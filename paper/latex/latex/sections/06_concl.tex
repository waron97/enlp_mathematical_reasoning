\label{sec:concl}

In summary, this reproduction attempt failed to achieve the reported 92.5\% accuracy of MathPrompter on MultiArith, falling short at around 82.3\%. This is the best result this study achieved, however it is not unimaginable that further optimisations that didn't get a mention in the original paper would boost accuracy to the reported levels.

Aspects of MathPrompter that should interest developers but that were not discussed by the originals authors were also highlighted here, such as expected API calls per question and execution time. Publishing such results should enable implementers to make more sound configuration options, both for customer satisfaction, and to lessen the strain on their wallets. 

Using LLMs to solve mathematical problems, and to process numeric values in general, is an interesting and active field of research that still has avenues for growth. While early results are promising, one should keep in mind that MultiArith exercises are all of the sort that humans would expect to see in the first years of elementary school, therefore an above 90\% accuracy on it should be more the default than something to be in awe of. Perhaps, in the future, research could consider venturing into the area of more complex mathematical problems, as there might be more to gain there.